\documentclass[11pt]{article}
\usepackage{mathtools}
\usepackage{amssymb}
\usepackage{amsthm}
\usepackage{enumerate}
\usepackage{tikz}
% \usepackage{tikz-cd}
% \usepackage{float}
% \usepackage{wrapfig}
% \usepackage{listings}

    \title{%
    \textbf{rummibot} \\
    \large On Computer Rummikub\\ and Game Complexity}

    \author{gsobell}
    \date{}
    \addtolength{\topmargin}{-3cm}
    \addtolength{\textheight}{3cm}
\begin{document}
\maketitle
\pagenumbering{arabic}

\begin{abstract}
Just some notes and thoughts from develepment.
In the future, I might include pseudocode, if I port to another language. Find a mistake? Let me know, open a pull request!
\end{abstract}
The following sections have to deal with how different scenarios affect game tree branching and complexity.

% $$\text{Ways to choose 14 tiles: } \binom{106}{14} \approx 1.0571 \times 10^{17}  $$

\section{Jokers}
There are two Jokers in the deck. Each Joker can assume any of the 52 unique tiles. If your opening hand had a joker, that means there are 52 hands to check. If you had both jokers, that increases the number of possible hands to check $52^2 =  2704$.
% Probability of choosing a joker:
% \begin{align*}
% \frac{1}{106} + \frac{1}{103} + \cdots + \frac{1}{92} =\frac{92}{}
% \end{align*}


\section{Sets}
Sets with 3 elements don't branch the game tree

\subsection{Groups}
Groups can either be $n = 3$ or $n = 4$ long.
Each group with 4 tiles creates 4 more options in the game tree.

\subsection{Runs}
Runs can be $3 \le n \le 13$ long. Each run with $n \ge 3$ can be split multiple ways.
For a run of length $n$, any unique partitining that make a sub-run of 3 or more adds a case to searched.
% do the math here
Thankfully, larger runs are excedingly rare, so it almost always will not affect runtime.






\end{document}
